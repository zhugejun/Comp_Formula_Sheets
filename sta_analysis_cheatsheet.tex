\documentclass[10pt,landscape]{article}
\usepackage{multicol}
\usepackage{calc}
\usepackage{ifthen}
\usepackage[landscape]{geometry}
\usepackage{graphicx}
\usepackage{amsmath, amssymb, amsthm}
\usepackage{latexsym, marvosym}
\usepackage{pifont}
\usepackage[sc]{mathpazo} % use mathpazo for math fonts
\usepackage{lscape}
\usepackage{graphicx}
\usepackage{array}
\usepackage{booktabs}
\usepackage[bottom]{footmisc}
\usepackage{tikz}
\usetikzlibrary{shapes}
\usepackage{pdfpages}
\usepackage{wrapfig}
\usepackage{enumitem}
\setlist[description]{leftmargin=0pt}
\usepackage{xfrac}
\usepackage[pdftex,
            pdfauthor={Gejun Zhu},
            pdftitle={Statistical Analysis Cheatsheet},
            pdfsubject={One page both sides cheatsheet for statistics analysis comprehensive exam.},
            pdfkeywords= {statistics} {cheatsheet} {pdf} {cheat} {sheet} {formulas}{equations}
            ]{hyperref}
\usepackage{relsize}
\usepackage{rotating}

 \newcommand\independent{\protect\mathpalette{\protect\independenT}{\perp}}
    \def\independenT#1#2{\mathrel{\setbox0\hbox{$#1#2$}%
    \copy0\kern-\wd0\mkern4mu\box0}} 
            
\newcommand{\noin}{\noindent}    
\newcommand{\logit}{\textrm{logit}} 
\newcommand{\var}{\textrm{Var}}
\newcommand{\cov}{\textrm{Cov}} 
\newcommand{\corr}{\textrm{Corr}} 
\newcommand{\N}{\mathcal{N}}
\newcommand{\Bern}{\textrm{Bern}}
\newcommand{\Bin}{\textrm{Bin}}
\newcommand{\Beta}{\textrm{Beta}}
\newcommand{\Gam}{\textrm{Gamma}}
\newcommand{\Expo}{\textrm{Expo}}
\newcommand{\Pois}{\textrm{Pois}}
\newcommand{\Unif}{\textrm{Unif}}
\newcommand{\Geom}{\textrm{Geom}}
\newcommand{\NBin}{\textrm{NBin}}
\newcommand{\Hypergeometric}{\textrm{HGeom}}
\newcommand{\Mult}{\textrm{Mult}}



\geometry{top=.3in,left=.2in,right=.2in,bottom=.3in}

\pagestyle{empty}
\makeatletter
\renewcommand{\section}{\@startsection{section}{1}{0mm}%
                                {-1ex plus -.5ex minus -.2ex}%
                                {0.5ex plus .2ex}%x
                                {\normalfont\large\bfseries}}
\renewcommand{\subsection}{\@startsection{subsection}{2}{0mm}%
                                {-1explus -.5ex minus -.2ex}%
                                {0.5ex plus .2ex}%
                                {\normalfont\normalsize\bfseries}}
\renewcommand{\subsubsection}{\@startsection{subsubsection}{3}{0mm}%
                                {-1ex plus -.5ex minus -.2ex}%
                                {1ex plus .2ex}%
                                {\normalfont\small\bfseries}}
\makeatother

\setcounter{secnumdepth}{0}

\setlength{\parindent}{0pt}
\setlength{\parskip}{0pt plus 0.5ex}

% -----------------------------------------------------------------------

\begin{document}

\raggedright
\footnotesize
\begin{multicols}{3}


% multicol parameters
% These lengths are set only within the two main columns
%\setlength{\columnseprule}{0.25pt}
\setlength{\premulticols}{1pt}
\setlength{\postmulticols}{1pt}
\setlength{\multicolsep}{1pt}
\setlength{\columnsep}{2pt}

%%%%%%%%%%%%%%%%%%%%%%%%%%%%%%%%%%%%
%%% TITLE
%%%%%%%%%%%%%%%%%%%%%%%%%%%%%%%%%%%%

\begin{center}
     \Large{\textbf{Statistical Analysis Cheatsheet}} \\
\end{center}

%%%%%%%%%%%%%%%%%%%%%%%%%%%%%%%%%%%%
%%% ATTRIBUTIONS
%%%%%%%%%%%%%%%%%%%%%%%%%%%%%%%%%%%%

\scriptsize

Compiled by Gejun Zhu (zhug3@miamioh.edu) in preparation for the analysis comprehensive exam by using William Chen's \href{http://wzchen.com/probability-cheatsheet}{formula sheet template}. 

\begin{center}
    Last Updated \today
\end{center}

% Cheatsheet format from
% http://www.stdout.org/$\sim$winston/latex/

%%%%%%%%%%%%%%%%%%%%%%%%%%%%%%%%%%%%
%%% BEGIN CHEATSHEET
%%%%%%%%%%%%%%%%%%%%%%%%%%%%%%%%%%%%


%%%%%%%%%%%%%%%%%%%%%%%
%%%%           Regression          %%%%
%%%%%%%%%%%%%%%%%%%%%%%
%
%           \begin{align*} 
%        ({\bf A} \cup {\bf B})^c \equiv {\bf A^c} \cap {\bf B^c} \\
%        ({\bf A} \cap {\bf B})^c \equiv {\bf A^c} \cup {\bf B^c}
%           \end{align*} 


\section{Regression Analysis}\smallskip \hrule height 2pt \smallskip

\subsection{Simple Linear Regression}
    \begin{description}
        \item[Model]: $Y_i = \beta_0 + \beta_1X_i + \epsilon_i$, $\epsilon_i \overset{iid}{\sim} N(0, \sigma^2)$, $Y_i\sim	N(\beta_0 + \beta_1X_i, \sigma^2)$. 
        %\item[Properties of SLR]
    \end{description}
    $E_i = Y_i - \hat{Y}_i$, $\sum E_i = 0$, $\sum Y_i = \sum \hat{Y}_i$, $\sum X_iE_i = 0$, $\sum \hat{Y}_iE_i = 0$. \\
    $B_1 = \frac{\sum_{i=1}^n (X_i-\bar{X})(Y_i-\bar{Y})}{\sum_{i=1}^n (X_i-\bar{X})^2} = \frac{S_{xy}}{S_{xx}}$, $S_{xy} = \sum XY - \frac{\sum X \sum Y}{n}$, $B_0 = \bar{Y} - B_1\bar{X}$. \\
    $B_1 = \sum_{i=1}^n Y_i \frac{(X_i-\bar{X})}{\sum_{i=1}^n (X_i-\bar{X})^2} =\sum_{i=1}^n k_i Y_i $; $\sum	k_i=0$, $\sum k_ix_i=1$, $\sum	 k_i^2 = \frac{1}{S_{xx}}$. \\
    $B_1 \sim	N(\beta_1, \frac{\sigma^2}{S_{xx}})$; $\frac{B_1 - \beta_1}{\sqrt{V(B_1)}}\sim N(0,1)$, $\frac{S_{B_1}^2}{V(B_1)} = \frac{MSE/S_{xx}}{\sigma^2/S_{xx}} = \frac{SSE}{(n-2)\sigma^2}$, \\
    $\frac{SSE}{\sigma^2}\sim	\chi_{n-2}^2$ $\Rightarrow$ $\frac{B_1 - \beta_1}{S_{B_1}} = \frac{B_1 - \beta_1}{\sqrt{MSE/S_{xx}}} \sim T_{n-2}$, CI: $b_1 \pm t_{1-\alpha/2, n-2} \cdot s_{b_1}$; $\frac{SSR}{\sigma^2} \sim	\chi_{p-1}^2$. \\
    Inference on $E(Y_h)$: $\hat{Y}_h = B_0 + B_1X_h $, $\hat{Y}_h \sim N(\beta_0 + \beta_1 X_h, \sigma^2[\frac{1}{n} + \frac{(X_h - \bar{X})^2}{S_{xx}}])$ \\
    Prediction on a new observation: $\hat{y} \pm t_{1-\alpha/2, n-2}\sqrt{mse[1+\frac{1}{n}+\frac{(X_h - \bar{X})^2}{X_{xx}}]}$. \\
    $SST = \sum	(Y_i - \bar{Y}_i)^2 = \sum	(Y_i - \hat{Y}_i)^2 + \sum	(\hat{Y}_i - \bar{Y}_i)^2 = SSE + SSR$\\
    If $var(Y_i) = \sigma^2$, and $Y_i's$ are uncorrelated, then $Cov(\sum	a_iY_i, \sum	b_iY_i) = \sigma^2\sum a_ib_i$. \\
    $B_1$ and $\bar{Y}$ are uncorrelated, $Cov(B_1, \bar{Y}) = 0$ because $Cov(B_1, \bar{Y}) = Cov(\sum k_i Y_i, \sum \frac{1}{n}Y_i) = \frac{\sigma^2}{n} \sum	k_i = 0$. \\
    Confidence intervals tell you about how well you have determined the mean. Prediction intervals tell you where you can expect to see the next data point sampled. \\
     %
    
    \begin{description}
    \item[ANOVA Table] - Analysis of variance for simple linear regression \\
        %\begin{table}[H]
        \begin{center}
            \begin{tabular}{r|cccc}
                \textbf{Source}  & \textbf{SS} & \textbf{DF} & \textbf{MS} &\textbf{ Expected}  \\ \hline
                \textbf{Regression} & SSR & 1 & MSR=SSR/1 & $\sigma^2 + \beta_1^2S_{xx}$\\
                \textbf{Error} & SSE & n-2 & MSE=SSE/(n-2) & $\sigma^2$ \\
                \textbf{Total} & SST &  n-1 &\  & \  \\
            \end{tabular}
        \end{center}
\end{description}
    Under $H_0: \beta_1=0$, $F^* = \frac{MSR}{MSE}\sim F_{1,n-2}$. $R^2 = \frac{SSR}{SST} = 1-\frac{SSE}{SST}$. $B_1 = r\sqrt{\frac{S_{yy}}{S_{xx}}}$. \\
    E(MSR) = $\sigma^2 + \beta_1^2S_{xx}$, SSR = $B_1^2S_{xx}$, $E(MSE) = E(\frac{SSE}{n-2}) = \frac{\sigma^2}{n-1}E(\frac{SSE}{\sigma^2}) = \sigma^2$\\
    Studentized residuals: $E_i^* = \frac{E_i}{\sqrt{\hat{V(E_i)}}}$ 
    
%%% assumptions ...    
    \begin{description}
    \item[Assumptions]  - LINE
    \begin{itemize}
    		\item Linearity: No curvature in the residual plot; (high-order, log/square)
    		\item Independence: $\epsilon_i \overset{iid}{\sim} N(0, \sigma^2)$;
    		\item Normality of error: QQ plot; (GLM, poisson regression...)
    		\item Equal Variance: standardized residual inside [-3, 3]. (weighting obs)
    \end{itemize}
    \end{description}


%%% Simultaneous Inference and other topics
%	\begin{description}
%	\item[Simultaneous Inference] - Simultaneous Inference and Other Topics \\
%	Bonferroni (family-wise): multiple p-values by g, the number of tests. \\ 
%	Holm (family-wise): compare smallest p-value with $\alpha /g$; second smallest with $\alpha / (g-1)$; ...\\
%	Regression through Origin: $B_1 = \frac{\sum Y_iX_i}{\sum X_i^2}$
%	\end{description}	    
    
    
    
    \begin{description}
    		\item[Matrix Approach] - Matrix form \\ %  \boldsymbol{}
    		$\mathbf{Y_{n\times 1}} = X_{n\times p}\boldsymbol{\beta}_{p\times 1}+\boldsymbol{\epsilon}_{n\times 1}$, $\boldsymbol{\epsilon}\sim	MN(\textbf{0}, \sigma^2\textbf{I})$. $\boldsymbol{\beta} = (X'X)^{-1}X'\boldsymbol{Y}$. \\
    		$\boldsymbol{Y} \sim MN(\mu, \sum)$, then $A\boldsymbol{Y} + \boldsymbol{b} \sim MN(A\mu+b, A\sum A')$. \\
    		$\boldsymbol{B}\sim \boldsymbol{MN}(\boldsymbol{\beta}, \sigma^2(X'X)^{-1})$, $\boldsymbol{\hat{Y}} = X\textbf{B} = H\boldsymbol{Y}\sim MN(X\boldsymbol{\beta}, \sigma^2H)$, $H=X(X'X)^{-1}X'$.\\
    		$S^2(\boldsymbol{B}) = MSE(X'X)^{-1}$; $\boldsymbol{E} = \boldsymbol{Y} - \boldsymbol{\hat{Y}} = (I - H)\boldsymbol{Y} \sim	MN(\boldsymbol{0}, \sigma^2(I-H))$.\\
    		$\sum Y_i^2 = \boldsymbol{Y}'J\boldsymbol{Y}$, $SSTO = \boldsymbol{Y}'(I - \frac{1}{n}J)\boldsymbol{Y}$, $SSE = \boldsymbol{Y}'(I-H)\boldsymbol{Y}$, $SSR = \boldsymbol{Y}'(H-\frac{1}{n}J)\boldsymbol{Y}$.\\
    		$\hat{Y} = HY$, $\hat{Y}' = Y'H$, $0 = \sum \hat{Y}_iE_i = \sum \hat{Y}_i Y_i - \sum \hat{Y}_i^2$. \\
    		Distribution of $\hat{Y}_h$: $\hat{Y}_h = \boldsymbol{X_h'B}\sim MN(\boldsymbol{X_h'\beta}, \sigma^2\boldsymbol{X_h'}(X'X)^{-1}\boldsymbol{X_h})$ \\
    		$S^2(\hat{Y_h}) = MSE(\boldsymbol{X}_h'(X'X)^{-1}\boldsymbol{X}_h)$ \\
    		$pred = Y_{h(new)} - \hat{Y}_h$, $pred \sim N(0, \sigma^2(1 + \boldsymbol{X}_{h(new)}'(X'X)^{-1}\boldsymbol{X}_{h(new)}))$
    		
    \end{description}
    
%%%%%%%%%%%%%%%    
    \begin{description}
    		\item[Multiple Regression] - Multiple regression \\ %  \boldsymbol{}
    		$\boldsymbol{Y} = X\beta+\boldsymbol{\epsilon}$, where $\epsilon \sim MN(\boldsymbol{0}, \sigma^2I)$. \\    	
    		$df(SSE) = n-p$, where $p$ is the number of parameters; $df(SSTo) = n-1$, $df(SSR) = p-1$.  Reject $H_0$ if $f^* =  \frac{MSR}{MSE} > F_{1-\alpha; p-1, n-p}$. $R^2 = 1-\dfrac{SSE}{SSTo}$, adjusted $R^2 = 1 - (\frac{n-1}{n-p}) \frac{SSE}{SSTo}$\\
    		CI: $b_1 \pm t_{1-\alpha/2, n-p} \cdot s_{b_1}$, Bonferroni CI:  $b_1 \pm t_{1-\alpha/2g, n-p} \cdot s_{b_1}$\\	
		\underline{\textit{FWER}} - Bonferroni and Holm. Bonferroni: compare p-value with $\alpha / g$, CL = $1-\alpha / g$, where $g$ is the number of tests; Holm: sort p-values and multiple g, g-1, ..., 1 in order and compare with $\alpha$ finally. \\
		%Bonferroni Corrected (1 - $\alpha$) x 100 \% Confidence Intervals: 
		$SSTo = SSR(X_1) + SSE(X_1) = SSR(X_1, X_2) + SSE(X_1, X_2)$, where $SSR(X_1, X_2) = SSR(X_1) + SSR(X_2|X_1)$, $SSE(X_1,  X_2) = SSE(X_2) - SSR(X_1|X_2)$.\\
		In general,  $SSR(X_q, X_{q+1},..., X_{p-1}|X_1, X_2,...,X_{q-1}) = SSE(X_1, X_2,...,X_{q-1}) - SSE(X_1, X_2, ...,X_{p-1}) = SSE_R - SSE_F$.\\
		\underline{\textit{Partial F-test (Reduced vs. Full)}}: $H_0: \beta_q = \beta_{q+1} = ... = \beta_{p-1} = 0$, $H_a$: At least one $\beta_k \neq 0,\ k = q, q+1, .., p-1$. Test statistic $F^* = \dfrac{\frac{SSE_R - SSE_F}{df_R - df_F}}{MSE_F} = \dfrac{\frac{SSR(X_q, X_{q+1}, ..., X_{p-1} | X_1, X_2, ..., X_{q-1})}{p-q}}{\frac{SSE(X_1, X_2, ..., X_{p-1})}{n-p}} $. If $H_0$ is true, then $F^* \sim F_{p-q, n-p}$.\\
		\underline{\textit{coefficient of partial determination}} is the proportion of the variation in Y "explained" by an indep. variable when other indep. variables are in the model. \\
		$R_{Y1|2}^2 = \frac{SSE(X_2) - SSE(X_1, X_2)}{SSE(X_2)} = \frac{SSR(X_1|X_2)}{SSE(X_2)} = 1 - \frac{SSE(X_1, X_2)}{SSE(X_2)}$ \\
		$R_{Y1|2,3,4} = 1 - \frac{SSE(X_1, X_2, X_3, X_4)}{SSE(X_2, X_3, X_4)}$\\
		\underline{\textit{multicollinearity diagnostic}}: Variance Inflation Factor (VIF) = $(1 - R_k^2)^{-1}$, where $R_k^2=$ coefficient of determination when $X_k$ is regressed upon other predictors. If $VIF>1$, variance of $B_k$ is inflated due to correlations b/w $X_k$ and other predictors. 	If $X_k$ is uncorrelated with other predictors, then $R_k^2 = 0$ and $VIF_k = 1$. \\ 
		
		
    		
    \end{description}    
    
    
%%%%%%%%%%%%%%%%
%      \underline{\textit{}}
\begin{description}
	\item[Model Diagnostics] - More about model diagnostics\\
	\underline{\textit{Added-variable Plots}} (1) regress Y on predictors except $X_k$ and obtain the residuals; (2) regress $X_k$ on other predictors and obtain residuals; (3) plot (1) vs (2);\\
	\underline{\textit{Leverage}}: A measure of how unusual an X is. (diagonal values of Hat matrix, $\sum h_{ii} = tr(H) = tr[X(X'X)^{-1}X'] = tr[X'X(X'X)^{-1}] = tr[I_{p\times p}] = p$) \\
	\underline{\textit{Influence}}: An influence point is its exclusive causes substantial changes to the fitted data. Just because a point has high leverage doesn't mean it has high influence. \\
	Measures of influence include: \\
	$DFFITS_i = \frac{\hat{Y}_i - \hat{Y}_i(i)}{\sqrt{MSE_{(i)}h_{ii}}}$ - a measure of an observation on its own fitted value. \\
	Cook's Distance = $\frac{e_i^2}{pMSE}[\frac{h_{ii}}{(1-h_{ii})^2}]$ - a measure of influence of observation $i$ on all the fitted value. \\
	$DFBETAS_{k(i)} = \frac{b_k - b_{k(i)}}{\sqrt{MSE_{(i)}c_{kk}}}$ where $c_{kk}$ is  $(k+1)^{th}$ diagonal in $(X'X)^{-1}$ -  a measure of the influence of observation $i$ on the parameter estimate $b_k$. It measures the difference b/w the parameter with / without observation $i$. 
	
\end{description}
    
%%%%%%%%%%%%%%%%%%%%%%%
%%%% Design of experiments   %%%
%%%%%%%%%%%%%%%%%%%%%%%

\section{Design of experiments} \smallskip \hrule height 2pt \smallskip

\subsection{CRD with one factor}
\begin{description}
    \item[Model] one factor with $a\geq 2$ levels. $H_0: \mu_1 = \mu_2, ..., = \mu_a$ or $\hat{\tau_i} = 0$.
    \begin{itemize}
         \item $y_{ij} = \mu + \tau_i + \epsilon_{ij}, i=1,2,...,a, j=1,2,...,n_j$, $\epsilon_{ij}\sim N(0,\sigma^2)$;
         \item $y_{i.} = \sum_j y_{ij}$, $E(y_{ij}) = \mu_i$, $var(y_{ij}) = \sigma^2$;
    		\item LSE estimator $\hat{\mu} + \hat{\tau}_i = \bar{y}_{i.}$,  if $\sum n_i\hat{\tau}_i = 0$ or $\hat{\mu = 0}$ or $\hat{\tau}_a=0$;
    		\item $MS_{trt} = \frac{SS_{trt}}{a-1} =\frac{\sum_{i=1}^a n_i (\bar{y}_i - \bar{y}_{..})^2}{a-1}$, $MS_E = \frac{SSE}{N-a} = \frac{\sum_{i=1}^a \sum_{j=1}^{n_i} (y_{ij} - \bar{y}_{i.})}{N-a}$;
    		\item $E(MS_E)=\sigma^2$, $E(MS_{trt}) = \sigma^2 + \frac{\sum_{i=1}^a n_i \tau_i^2}{a-1}$, $S_p^2 = \frac{(n_1-1)S_1^2 + (n_2-1)S_2^2}{n_1+n_2-2}$;
    		\item $SS_T = \sum_{i=1}^a \sum_{j=1}^{n_i} y_{ij}^2 - \frac{(y_{..})^2}{N}$, $SS_{trt} = \sum_{i=1}^a \frac{y_i^2}{n_i} - \frac{(y_{..})^2}{N}$;
    		\item Fact: Under $H_0$, $SSE/\sigma^2 \sim \chi_{N-a}^2$, $SS_{trt}/\sigma^2 \sim \chi_{a-1}^2$, independent;
    		\item $\frac{SS_{trt}/(a-1)\sigma^2}{SS_{E}/(N-a)\sigma^2} \sim F_{a-1, N-a}$; rej  $F_0 > F(\alpha, a-1, N-a)$,  $p = P(F_{a-1, N-a} > F_0)$;
    		\item $E(\bar{y}_{i.}) = \mu_i$, $V(\bar{y}_{i.})=\sigma^2/n_i$, $\frac{\bar{y}_{i.} - \mu_i}{\sqrt{MSE/n_i}}\sim T_{N-a}$;
    		\item CI: $\bar{y}_{i.} \pm t_{\alpha /2, N-a}\sqrt{MSE/n_i}$, $\bar{y}_{s.} - \bar{y}_{t.} \pm t_{\alpha /2, N-a}\sqrt{MSE/n_s+MSE/n_t}$.
    		\item Linear contrasts: $\Gamma = \sum c_i\mu_i$, $C = \sum c_i\hat{y}_{i.}$ with $\sum c_i = 0$. $E(C)=\Gamma$, $V(C) = \sigma^2\sum \frac{c_i^2}{n_i}$. CI: $\sum c_i\hat{y}_{i.} \pm t_{\alpha /2, N-a}\sqrt{MSE\sum \frac{c_i^2}{n_i}}$, $t=\frac{\sum c_i\hat{y}_{i.} - c}{\sqrt{MSE\sum \frac{c_i^2}{n_i}}}$.
    \end{itemize}
    \end{description}
    
    \begin{description}
    \item[ANOVA Table] Analysis of variance for three factor fixed effects model. \\
        %\begin{table}[H]
        \begin{center}
            \begin{tabular}{r|cc}
                \textbf{Source}  & \textbf{DF} & \textbf{Expected Mean Square}  \\ \hline
                \textbf{A} & $a-1$ & $\sigma^2+\frac{bcn\sum\tau_i^2}{a-1}$ \\
                \textbf{AB} & $(a-1)(b-1)$ & $\sigma^2+\frac{cn\sum\sum(\tau\beta)_{ij}^2}{(a-1)(b-1)}$ \\
                \textbf{ABC} & $(a-1)(b-1)(c-1)$ & $\sigma^2+\frac{n\sum\sum\sum(\tau\beta\gamma)_{ijk}^2}{(a-1)(b-1)(c-1)}$ \\
                \textbf{Error} & $abc(n-1)$ & $\sigma^2$ \\
            \end{tabular}
        \end{center}
    
\end{description}


\subsection{Basic Blocking Designs}
	\begin{description}
		\item[Model] two factors - the treatment factor $\tau_i$ and the block factor $\beta_j$. \\
		$Y_{ij} = \mu + \tau_i + \beta	_j + \epsilon_{ij}, i = 1,2,...,a, j = 12,...,b$. ($\sum \tau_i = 0$ and $\sum \beta_j = 0$)\\
		$H_0: \tau_0 = \tau_1 = ... = \tau_a = 0$ or $\mu_1 = \mu_2 = ... = \mu_a$, $H_a: Not\ H_0$ (at least two means differs) . $E(\bar{Y_{i.}}) = \mu + \tau_i$\\%, $df(Blocks) = b-1$, $df(Error) = N-a-b+1$ \\ 
		%$SS_{T} = \sum \sum y_{ij}^2 - \frac{y_{..}^2}{N}$, $SS_{trt} = \sum \frac{y_{i.}^2}{b} - \frac{y_{..}^2}{N}$, $SS_{blk} = \sum \frac{y_{.j}^2}{a} - \frac{y_{..}^2}{N}$ \\
		%$E(MS_{trt}) = \sigma^2 + \frac{b\sum \tau_i^2}{a-1}$, $E(MS_{blk}) = \sigma^2 + \frac{a \sum \beta_j^2}{b-1}$, $E(MSE) = \sigma^2$\\
		%$F_0 = MSE/MS_{trt}$, p-value = $P(F_{a-1, (a-1)(b-1)} > F_0)$.\\
		A balanced incomplete block design (BIBD) includes a treatment factor with $a$ levels, a blocking factor with  $b$ levels, each block includes $k$ experimental units, which implies a total of $bk$ runs. This means that each treatment occurs $r = bk/a$ times. Each treatment occurs either 0 or 1 times, and each pair of treatments occurs together in a block exactly $\lambda$ times.  $N = bk$. (1) $ar = bk$; (2) $r(k-1) = \lambda (a-1)$; (3) $b\geq a$. \\
		    \begin{center}
            \begin{tabular}{r|cc}
                \textbf{Source}  & \textbf{DF} & \textbf{Sum of Squares}  \\ \hline
                \textbf{Treatments} & $a-1$ & $\sum_i \frac{y_{i.}^2}{b} - \frac{y_{..}^2}{N}$ \\
                \textbf{Blocks} & $b-1$ & $\sum_j \frac{y_{.j}^2}{a} - \frac{y_{..}^2}{N}$ \\
                \textbf{Error} & $N-a-b+1$ & $SS_{total} - SS_{trts} - SS_{blocks}$ \\
                \hline
                \textbf{Total} & $N-1$ & $\sum \sum y_{ij}^2 - \frac{y_{..}^2}{N}$ \\
            \end{tabular}
        \end{center}
        $E(MS_{trt}) = \sigma^2 + \frac{b\sum \tau_i^2}{a-1}$, $E(MS_{blk}) = \sigma^2 + \frac{a \sum \beta_j^2}{b-1}$, $E(MSE) = \sigma^2$\\
        $F_0 = MS_{trt}/MSE$, p-value = $P(F_{a-1, (a-1)(b-1)} > F_0)$.\\
        $Q_i = y_{i.} - \dfrac{1}{k} \sum_j n_{ij}y_{.j}$, $\hat{\tau_i} = \dfrac{kQ_i}{\lambda a}$, $\hat{\mu} = \dfrac{y_{..}}{N} = \dfrac{y_{..}}{bk}$, $LSMean(\mu_i) = \hat{\mu} + \hat{\tau}_i$
	\end{description}
	
	
\subsection{Factorial Designs}
	\begin{description}
	\item[Model] $Y_{ijk} = \mu + \tau_i + \beta_j + (\tau \beta)_{ij} + \epsilon_{ijk}, \ i = 1,2,...,a, j = 1,2,..., b, k = 1,2, ..., n$\\
	$\sum \tau = 0$, $\sum	\beta = 0$, $\sum_i (\tau \beta)_{ij}  = 0$, $\sum_j (\tau \beta)_{ij}  = 0$ \\ 
	$\hat{\mu} = \bar{y}_{...}$, $\hat{\tau}_i = \bar{y}_{i..} - \bar{y}_{...}$, $\hat{\beta}_j = \bar{y}_{.j.} - \bar{y}_{...}$, $\hat{\tau \beta}_{ij} = \bar{y}_{ij.} - \bar{y}_{i..} - \bar{y}_{.j.} + \bar{y}_{...}$\\
	$\mu_{i.} = \mu + \tau_i = $mean of $ith$ level of A; $\mu_{.j} = \mu + \beta_j = $mean of $jth$ level of B; $\mu_{ij} = \mu + \tau_i + \beta_j + (\mu \beta)_{ij}= $mean of $ijth$ treatment. \\ 
	$E(\hat{\mu}) = E(\bar{Y}_{...}) = \mu$, $var(\hat{\mu}) = \frac{\sigma^2}{abn}$; \\ 
		   \begin{center}
            \begin{tabular}{r|cc}
                \textbf{Source}  & \textbf{DF} & \textbf{Sum of squares}  \\ 
                \hline
                \textbf{Treatments} & $ab-1$ & $\frac{1}{n} \sum_i \sum_j y_{ij.}^2 - \frac{(y_{...})^2}{abn}$ \\
                \textbf{A} & $a-1$ & $\frac{1}{bn} \sum_i y_{i..}^2 - \frac{(y_{...})^2}{abn}$ \\
                \textbf{B} & $b-1$ & $\frac{1}{an} \sum_j y_{.j.}^2 - \dfrac{(y_{...})^2}{abn}$ \\
                \textbf{AB} & $(a-1)(b-1)$ & $SS_T - SS_A - SS_B$ \\
                \textbf{Error} & $ab(n-1)$ & $SS_T - SS_{trts}  = \sum_i \sum_j \sum_k (y_{ijk} - \bar{y}_{ij.})^2$ \\
                \hline
                \textbf{Total} & $abn-1$ & $\sum_i \sum_j \sum_k y_{ijk}^2 - \frac{(y_{...})^2}{abn}$ \\
            \end{tabular}
        \end{center}	
	Overall test: $\mu_{11} = \mu_{12} = ... = \mu_{ab}$, test statistic $F_0 = \frac{MS_{trt}}{MSE}$; \\
	Interaction test: $(\alpha\beta)_{ij} = 0$ for all $ij$, test statistic $F_0 = \frac{MS_{AB}}{MSE}$.\\
	sample size: $\delta^2 = \frac{nb\Delta^2}{2\sigma^2}$, assuming 2 levels of A differ by $\Delta$; $\delta^2 = \frac{n\Delta^2}{2\sigma^2}$ from the interaction terms. 
	\end{description}
	
\subsection{$2^k$ Factorial Designs}

\subsection{Two-level Fractional Factorial Designs}
\begin{description}
\item[Design resolution]- A fractional factorial design's resolution is the length of the shortest word and its defining relation. \\
$2^{k-p}$ terms, $2^p$ alias.\\

\end{description}

%\subsection{RSM}

\subsection{Random Effects and Mixed Models}
\begin{description}
\item[Model] $Y_{ij} = \mu + \tau_i + \epsilon_{ij},\ i = 1,2,...,a; j = 1, 2,..., n_i$, where $\tau_i$  are assumed to be independent $N(0, \sigma_{\tau}^2)$ random variables. \\ 
$H_0: \sigma_{\tau}^2 = 0$ vs. $H_a: \sigma_{\tau}^2 > 0$, test stat: $F_0 = \frac{MS_{trt}}{MSE}$, $F_0 \sim F_{a-1,N-a}$ under $H_0$. \\
\underline{Some facts}: $Y_{ij} \sim N(\mu , \sigma_{\tau}^2+\sigma^2)$ (1) if $i\neq k$ - different treatment levels, $Cov(Y_{ij}, Y_{kj}) = 0$ since $\tau_i$ and $\tau_k$ are independent and $E(\tau_i\tau_k) = E(\tau_i)E(\tau_k) = 0$; (2) if $k \neq l$ - same treatment different obs, $Cov(Y_{ij}, Y_{kj}) = \sigma_{\tau}^2$. \\
\underline{two-way random model}: $Cov(Y_{ijk}, Y_{ijk'}) = \sigma_{\tau}^2 + \sigma_{\beta}^2 + \sigma_{\tau \beta}^2$ if $k \neq k'$; $Cov(Y_{ijk}, Y_{ij'k}) = \sigma_{\tau}^2$ if $j \neq j'$. \\
$E(MSE) = \sigma^2$, $E(MS_{trt}) = \sigma^2 + n_0\sigma_{\tau}^2$ where $n_0 = n$ if all $n_i = n$ and $n_0 = \frac{1}{a-1}[N-\frac{\sum n_i^2}{N}]$.\\
Estimates: $\hat{\sigma}^2 = MSE$ and $\hat{\sigma}_{\tau}^2 = \frac{MS_{trt} - MSE}{n_0}$.\\
Confidence interval for $\frac{\sigma_{\tau}^2}{\sigma_{\tau}^2 + \sigma^2}$: $\frac{MS_{trts} / (n\sigma_{\tau}^2 + \sigma^2)}{MS_E / \sigma^2} \sim F_{a-1,N-a}$, $(F_{1-\alpha/2, a-1, N-a} \leq \frac{MS_{trts}}{MS_E} \frac{\sigma^2}{n\sigma_{\tau}^2 + \sigma^2} \leq F_{\alpha /2, a-1, N-a}) = 1 - \alpha$, $P(L\leq \frac{\sigma_{\tau}^2}{\sigma^2} \leq U) = 1-\alpha$, $L = \frac{1}{n} (\frac{MS_{trts}}{MS_E} \frac{1}{F_{\alpha/2, a-1, N-a}} - 1)$, $U = \frac{1}{n} (\frac{MS_{trts}}{MS_E} \frac{1}{F_{1 - \alpha/2, a-1, N-a}} - 1)$, $\frac{L}{L+1} \leq \frac{\sigma_{\tau}^2}{\sigma_{\tau}^2 + \sigma^2} \leq \frac{U}{1+U}$\\
\underline{Two-factor factorial with random factors}: $Y_{ijk} = \mu + \tau_i + \beta_j + (\tau \beta)_{ij} + \epsilon_{ijk}$,  $i = 1,2,...,a$, $j = 1,2,...,b$, $k = 1,2,...,n$, $V(\tau_i) = \sigma_{\tau}^2$, $V(\beta_j) = \sigma_{\beta}^2$, $V[(\tau \beta)_{ij}] = \sigma_{\tau \beta}^2$, and $V(\epsilon) = \sigma^2$. \\
Expected mean squares: $E(MS_A) = \sigma^2 + n\sigma_{\tau \beta}^2 + bn\sigma_{\tau}^2$; $E(MS_B) = \sigma^2 + n\sigma_{\tau \beta}^2 + an\sigma_{\beta}^2$; $E(MS_{AB}) = \sigma^2 + n\sigma_{\tau \beta}^2$; $E(MSE) = \sigma^2$.\\
\underline{Two-factor mixed model}: Factor A is fixed; factor B is random. $Y_{ijk} = \mu + \tau_i + \beta_j + (\tau \beta)_{ij} + \epsilon_{ijk}$, where
\begin{itemize}
	\item $i = 1,2,...,a$, $j = 1,2,...,b$, $k = 1,2,...,n$;
	\item $\tau_i$ is a fixed effect with $\sum \tau_i = 0$;
	\item  $\beta_j \sim N(0, \sigma_\beta ^2)$, $(\tau\beta)_{ij} \sim N(0, \sigma_{\tau\beta}^2)$, and $\epsilon_{ijk} \sim N(0, \sigma^2)$.
\end{itemize}
 $Y_{ijk} \sim N(\mu + \tau_i, \sigma^2 + \sigma_\beta	^2 + \sigma_{\tau\beta}^2)$. \\
Expected mean squares: $E(MSE) = \sigma^2$, $E(MS_A) = \sigma^2 + n\sigma_{\tau\beta}^2 + bn\frac{\sum \tau_i}{a-1}$, $E(MS_B) = \sigma^2 + n\sigma_{\tau\beta}^2 + an\sigma_\beta^2$, $E(MS_{AB}) = \sigma^2 + n\sigma_{\tau\beta}^2$\\
Variance components estimates: $\hat{\sigma}^2 = MSE$, $\hat{\sigma}_{\tau\beta}^2 = \frac{MS_{AB} - MSE}{n}$, $\hat{\sigma}_{\beta}^2 = \frac{MS_{B} - MS_{AB}}{an}$ \\
Hypothesis Tests: (1) $H_0: \sigma_{\tau\beta}^2 = 0$ vs. $H_a: \sigma_{\tau\beta}^2 > 0$ using $F = \frac{MS_{AB}}{MSE}$; (2) $H_0: \sigma_{\beta}^2 = 0$ vs. $H_a: \sigma_{\beta}^2 > 0$ using $F = \frac{MS_{B}}{MS_{AB}}$; (3) $H_0: \tau_i = 0$ vs. $H_a: not\ H_0$ using $F = \frac{MS_{A}}{MS_{AB}}$. \\
Approximate F-test: degree of freedom $\nu = \frac{(\sum c_i MS_i)^2}{\sum \frac{c_i^2 MS_i^2}{\nu_i^2}}$
\end{description}

\subsection{Nested Designs}

\begin{description}
\item[Model] $Y_{ijk} = \mu + \tau_i + \beta_{j(i)} +  \epsilon_{ijk},\ i = 1,2,...,a; j = 1, 2,...,b, k= 1,2,...,n$. \\
Crossed vs. nested factors: Two factors are considered crossed if every level of one factor occurs with every level of the other factor. In two factor design, one factor is nested with another when the levels of one factor are different within each level of the other factor. \\
If fixed: $\sum \tau_i = 0$, $\sum \beta_{j(i)} = 0$ for all $i$.\\
If random: $\beta_{j(i)} \sim N(0, \sigma_{\beta}^2)$, $\tau_i \sim N(0, \sigma_{\tau}^2)$, independent;\\
If mixed ($\tau$ is fixed, and $\beta$ is random): $\sum \tau_i = 0$, $\beta_{j(i)} \sim N(0, \sigma_{\beta}^2)$, independent.\\
$SSE = \sum_i \sum_j \sum_k (Y_{ijk} - \bar{Y}_{ij.})^2$ with $df = N-ab$\\
$SS_{B(A)} = n\sum_i \sum_j (\bar{Y}_{ij.} - \bar{Y}_{i..})^2$ with $df = a(b-1)$ \\
A random, B(A) random: $Cov(Y_{ijk}, Y_{mno}) = \sigma_\beta^2 + \sigma_\tau^2$ if $i=m, j=n, k\neq o$; $Cov(Y_{ijk}, Y_{mno}) =  \sigma_\tau^2$ if $i=m, j\neq n$; $Cov(Y_{ijk}, Y_{mno}) = 0$ if $i\neq m$; \\
A fixed, B(A) random: $Cov(Y_{ijk}, Y_{mno}) = \sigma_\beta^2 $ if $i=m, j=n$; $0$ otherwise.
\end{description}

%%%%%%%%%%%%%%%%%%%%%%%
%%%%   General linear models   %%%
%%%%%%%%%%%%%%%%%%%%%%%
\section{Generalized Linear Models}\smallskip \hrule height 2pt \smallskip
\subsubsection*{Textbook 1}
An othognoal matrix $C_{k\times k}$ has the property $C'C=CC'=I$, i.e. $C'=C^{-1}$. The eigenvalues of $A_{k\times k}$ are the same as $C'AC$. \\
$P$ and $Q$ are nonsingular, then $rank\left(AQ\right)=rank\left(PA\right)=rank\left(A\right)$. \\
$A{}_{n\times n}$, symmetric, then $\textbf{x}_{i}'\textbf{x}_{j}=0$ for $i\ne j$. $P_{n\times n}$ nonsingular, then $Tr\left(P^{-1}AP\right)=Tr\left(A\right)$. \\
$A{}_{n\times n}$, symmetric, then $A$ can be factorized as $A=P\textbf{\ensuremath{\Lambda}}P^{-1}$, where $\textbf{\ensuremath{\Lambda}}_{ii}=\lambda_{i}$, $P$ is an orthogonal matrix, i.e. $PP'=I$.\\
$A{}_{n\times n}$, symmetric, idempotent, then $r\left(A\right)=tr\left(A\right)=r\left(P'AP\right)=tr\left(P'AP\right)$. \\
$z=a'Y$, $\frac{\partial z}{\partial Y}=a$; $z=Y'Y$, $\frac{\partial z}{\partial Y}=2Y$; $z=Y'AY$, $\frac{\partial z}{\partial Y}=AY+A'Y$.\\
$E\left(Y\right)=\mu$, $E\left(a'Y\right)=a'E\left(Y\right)=a'\mu$; $V\left(Y\right)=V$, $V\left(a'Y\right)=a'V\left(Y\right)a$, $V\left(AY\right)=AV\left(Y\right)A'$.\\
$E\left(Y'AY\right)=tr\left(AV\right)+\mu'A\mu$.\\
If $Y_{k\times1}\sim N\left(\mu,I\right)$, then $Y'Y\sim\chi_{k,\lambda=\frac{1}{2}\left(\mu'\mu\right)}^{2}$.\\
$Y_{n\times1}\sim N\left(\mu,I\right)$, $A=A'$, then $Y'AY\sim\chi_{k,\lambda}^{2}$ with $k=r\left(A\right),$ $\lambda=\frac{1}{2}\left(\mu'A\mu\right)$ iff $A=A^{2}$.\\
$Y_{n\times1}\sim N\left(\mu,\sigma^{2}I\right)$, $A=A'$, then $Y'AY\sim\chi_{k,\lambda}^{2}$ with $k=r\left(A\right),$ $\lambda=\frac{1}{2\sigma^{2}}\left(\mu'A\mu\right)$ iff $A=A^{2}$.\\
$Y_{n\times1}\sim N\left(\mu,V\right)$, $A=A'$, then $Y'AY\sim\chi_{k,\lambda}^{2}$ with $k=r\left(AV\right)=r\left(A\right),$ $\lambda=\frac{1}{2}\left(\mu'A\mu\right)$ iff $AV=\left(AV\right)^{2}$.\\
$Y_{n\times1}\sim N\left(\mu,V\right)$, then $Y'V^{-1}Y\sim\chi_{k,\lambda}^{2}$, with $k=n,\ \lambda=\frac{1}{2}\left(\mu'V^{-1}\mu\right)$.\\
$Y_{n\times1}\sim N\left(\mu,V\right)$, then $AY$ and $BY$ are independent iff $AVB'=0$.\\
 $Y_{n\times1}\sim N\left(\mu,V\right)$, $A_{n\times n}=A'$, $B_{m\times n}$, then $Y'AY$ and $BY$ are independent iff $BVA=0$.\\
 $Y_{n\times1}\sim N\left(\mu,V\right)$, $A_{n\times n}=A'$, $B_{n\times n}=B'$, then $Y'AY$ and $Y'BY$ are independent iff $AVB=0$.\\
 $B=\left(X'X\right)^{-1}XY$, $\hat{Y}=XB=X\left(X'X\right)^{-1}XY=HY$, $E\left(B\right)=\beta$, $var\left(B\right)=\sigma^{2}\left(X'X\right)^{-1}$, $E\left(\hat{Y}\right)=X\beta$, $var\left(\hat{Y}\right)=\sigma^{2}H$.\\
 $SSE=Y'\left(I-H\right)Y$ with $df=n-p$, $SSR=Y'\left(H-\frac{1}{n}J\right)Y$ with $df=p-1$ , $SST=Y'\left(I-\frac{1}{n}\right)Y$ with $df=n-1$. \\
 If $Y=X\beta+\epsilon$, $\epsilon\sim N\left(0,\sigma^{2}I\right)$, then $B=\left(X'X\right)^{-1}XY\sim N\left(\beta,\sigma^{2}\left(X'X\right)^{-1}\right)$. \\
 $\dfrac{\left(n-p\right)s^{2}}{\sigma^{2}}=\dfrac{\left(n-p\right)MSE}{\sigma^{2}}=\dfrac{SSE}{\sigma^{2}}=\dfrac{1}{\sigma^{2}}Y'\left(I-H\right)Y\sim\chi_{n-p}^{2}$.\\
 $B$ and $\dfrac{SSE}{\sigma^{2}}$ are independent. \\
 $\dfrac{b_{j}-\beta_{j}}{\sqrt{var\left(b_{j}\right)}}=\dfrac{b_{j}-\beta_{j}}{\sigma\sqrt{c_{jj}}}\sim N\left(0,1\right)$ ,$c_{jj}$ is $jth$ diag entry of $\left(X'X\right)^{-1}$.\\
 $\dfrac{\left(b_{j}-\beta_{j}\right)/\left(\sigma\sqrt{c_{jj}}\right)}{\sqrt{\frac{SSE}{\sigma^{2}}/\left(n-p\right)}}\sim t_{n-p}$ $\Rightarrow$ $b_{j}\pm t_{n-p}\sqrt{MSEc_{xx}}$\\
 $LB\sim N\left(L\beta,\sigma^{2}L\left(X'X\right)^{-1}L'\right)$.\\
 Let $M=\left(LB\right)'\left(\sigma^{2}\left(L\left(X'X\right)^{-1}L'\right)^{-1}\right)^{-1}\left(LB\right)\sim\chi_{r,\lambda}^{2}$, where $\lambda=\dfrac{1}{2\sigma^{2}}\left(LB\right)'\left(L\left(X'X\right)^{-1}L'\right)^{-1}\left(LB\right)$.\\
 $E\left(M\right)=r\sigma^{2}+\left(L\beta\right)'\left(L\left(X'X\right)^{-1}L'\right)^{-1}\left(L\beta\right)$\\
 $F^{*}=\dfrac{\left(Lb\right)'\left(L\left(X'X\right)^{-1}L'\right)^{-1}\left(Lb\right)/r}{\frac{SSE}{\sigma^{2}}/\left(n-p\right)}=\dfrac{MSQ}{MSE}\sim F_{r,n-p}$ under $H_{0}:L\beta=0$.\\
 $\dfrac{SSR}{\sigma^{2}}\sim\chi_{p,\lambda}^{2}$, where $\lambda=\dfrac{1}{2\sigma^{2}}\beta'\left(X'X\right)\beta$.\\
 $MSQ\left(L\beta\right)=\left(LB\right)'\left(\sigma^{2}\left(L\left(X'X\right)^{-1}L'\right)^{-1}\right)^{-1}\left(LB\right)=\dfrac{SSR}{\sigma^{2}}$.\\
 $A=X\left(X'X\right)^{-1}X'-X_{2}\left(X_{2}'X_{2}\right)^{-1}X_{2}'$ is idempotent; $r\left(A\right)=r$.\\

\subsubsection*{Textbook 2}
\vspace{-5pt}
standardized residual $r_{i}=\dfrac{y_{i}-\hat{\mu}_{i}}{\hat{\sigma}}$. \\
$f\left(y;\theta\right)=exp\left\{ a\left(y\right)b\left(\theta\right)+c\left(\theta\right)+d\left(y\right)\right\} $ \\
glm = exp family + link func (mono + diff)  \\
$E\left[a\left(y\right)\right]=-c'\left(\theta\right)/b\left(\theta\right)$ \\
$var\left[a\left(y\right)\right]=\dfrac{b''\left(\theta\right)c'\left(\theta\right)-c''\left(\theta\right)b'\left(\theta\right)}{\left[b'\left(\theta\right)\right]^{3}}$ \\
%$l(\theta; y) = a(y)b(\theta) + c(\theta) + d(y)$\\
score info: $U=\frac{\partial ln[f\left(\theta;y\right)]}{\partial\theta}=a\left(y\right)b'\left(\theta\right)+c'\left(\theta\right)$ \\
sampling dist of score statistics: $E\left(U\right)=0;$ $J=var\left(U\right)=-E\left(U'\right) = -E(\dfrac{\partial^2lnf(x;\theta)}{\partial \theta^2})=\dfrac{b''\left(\theta\right)c'\left(\theta\right)}{b'\left(\theta\right)}-c''\left(\theta\right)$ \\
One parameter: $\dfrac{U-0}{\sqrt{J}} \sim N\left(0,1\right)$, $U'J^{-1}U=\dfrac{U^{2}}{J}\sim\chi_{(1)}^{2}$; for a vector of parameters: $U\sim MVN(0, J)$, then $U'J^{-1}U\sim \chi_p^2$.\\
Approximate $l$ and $U$ with Taylor series: $l(\beta) = l(b) + (\beta - b)'U(b) - 1/2(\beta	- b)'J(b)(\beta	- b)$; $U(\beta	) = U(b) - J(b)(\beta - b)$. \\
sampling dist of MLE's: $U(b) = 0$, $J^{-1}U \sim N(0, J^{-1})$, $(b - \beta)'J(b)(b - \beta) \sim \chi_p^2$. \\
wald statistic $\left(b-\beta\right)'J\left(b\right)\left(b-\beta\right)\sim\chi^{2}\left(p\right)$\\
GLiM: (1) response follows independent exponential family dist; (2) predictors $X's$ available; (3) g(): monotone differentiable link function.\\
saturated model is a model with the maximum number of parameters that can be estimated. $\beta_{max}$ is the parameter for saturated model, $b_{max}$ is MLE for saturated model.	\\
$\lambda=\dfrac{L\left(b_{max};y\right)}{L\left(b;y\right)}$, $b_{max}$/b, - MLE for saturated/reduced model \\
Deviance: $D=2\left[l\left(b_{max};y\right)-l\left(b;y\right)\right]$ \\
$AIC=-2l\left(\hat{\pi};y\right)+2p$; $BIC=-2l\left(\hat{\pi};y\right)+2p\times\left(\#\text{of obs}\right)$ \\

\end{multicols}

\end{document}